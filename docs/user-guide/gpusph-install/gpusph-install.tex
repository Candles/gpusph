% vi:tw=72:fenc=utf-8
\documentclass{../GPUSPHtemplate}

\title{GPUSPH Installation Guide}

\author{}

\date{\currentver\ --- \reldate}

\begin{document}

\maketitle
\tableofcontents
\newpage

\section{Introduction}

GPUSPH is an implementation of Smoothed Particle Hydrodynamics (SPH) on
\nvidia\ CUDA-enabled graphics cards. The first version of GPUSPH was
developed by Alexis Hérault, guided by SPHysics, and presented at the
Third SPHERIC Workshop in Lausanne, Switzerland in 2008. 
The graphics processing unit (GPU) implementation came from GPU-LAVA, 
a lava flow program, developed by Hérault and Bilotta at INGV in 
Catania, Italy. The present version of GPUSPH is open source, 
licensed under the GNU General Public License
(\url{www.gnu.org/licenses/gpl.txt}).

Smoothed Particle Hydrodynamics (SPH) is a Lagrangian meshless numerical
method that was developed in astrophysics by \cite{lucy_numerical_1977} and
\cite{gingold_smoothed_1977}. Its first application to free surface flows (e.g.
dam breaks and waves) was by \cite{monaghan_volcanoes_1994}.
% \cite{gomez-gesteira_using_2004} and \cite{dalrymple_numerical_2006}, also
% applying SPH to dam breaks and waves, began the development of SPHysics,
% an open source FORTRAN code (\url{http://www.sphysics.org}),
% \cite{gomez-gesteira_sphysics_2012}.
Since in SPH the interactions between particles involve many neighbors
(several hundreds in three dimensions), it suffers from high computational costs.
This motivated the development of massively parallel SPH codes,
in particular codes running on graphics cards due to their performance and relatively
low cost.

The development of sophisticated graphics cards is driven by the demands of 
advanced computer gaming, in particular to handle three-dimensional 
graphics for the computer display. Each of these
graphics cards has numerous streaming processors to do the mathematics
of image rotation, resizing etc. With the advent of the {\em CUDA}
programing language from \nvidia\ in 2007, simple \cpp\ language can be used
to access the mathematical power of these massively parallel cards. For
computer simulations that are not data-intensive, GPU programming
provides supercomputer capabilities at commodity prices.

Some timing information can be found in \cite{herault_sph_2010}, 
showing that using the GPU is far faster (orders
of magnitude) than using a CPU to compute SPH models. Speedups of 100
can be achieved for parts of the code when compared to serial versions
of the code.

The first version of GPUSPH was running  on \nvidia's Compute Capability (CC) 1.x cards, 
GeForce 8xxx cards. From this first version we tested GPUSPH on all \nvidia\  architectures
(from Fermi CC 2.x to the latest Pascal CC 6.x). \nvidia\  dropped support for CC 1.x and will 
soon do it for CC 2.x. So at the moment GPUSPH will run on any card with CC 2.x or higher
(from Fermi up) but we expect to drop soon the support for CC 2.x. When done GPUSPH will
run on any card with CC 3.x or higher (from Kelpler up).

This guide is divided into several sections. 
First, the installation and set-up of the GPUSPH code is explained and 
some example problems to illustrate its use are provided.
The second chapter goes through all the steps necessary to build a new
simulation and post-process the results.
The third chapter deals with an overview of SPH, with which the reader should
have some familiarity. 
Finally we discuss the nature of the GPUSPH program in some detail.

\section{Installation of GPUSPH}

The first step to run GPUSPH is to install the \nvidia\ company's CUDA
compilers and libraries (directions given below). CUDA is an extension
of the \cpp\ language to allow \cpp\ to talk to the graphics card.

The second step is to install the open source software, CHRONO,
which simulates rigid body dynamics. This library is used for
any rigid objects that move, such as floating objects or objects moved
by fluid flow.

The third step is to obtain, compile and run GPUSPH.

\textbf{Remark:} to run multi-node simulations, you also need to
install OpenMP (>= 1.8.4).

\subsection{Installing CUDA}

Ensure that your computer has an \nvidia\ graphics card that is CUDA
enabled. The \nvidia\ website has a list of all the CUDA-enabled graphics
cards: \url{www.nvidia.com/object/cuda_gpus.html}. You can check whether
CUDA is already installed on your machine by launching the command:
\begin{shellcode}
nvidia-smi
\end{shellcode}
from the terminal. If CUDA is installed it will give you information
on the current state of the NVIDIA graphics card(s) on the machine.

\textbf{Note}: GPUSPH runs on cards with Compute Capabillity at least 2.0


\textbf{Remark}: regarding the choice of the GPU, the more CUDA 
cores and the more memory on the card, the better. Anyway a
mid range mobile GPU's like GT750m/GT840 with 1 or 2GB of memory
is sufficient to run significant simulations. A laptop with such a GPU will be
a perfect mobile developing and testing platform.

The GPU programming language CUDA can be obtained from the \nvidia\ website,
CUDA Zone. The CUDA Toolkit and CUDA Software Development Kit (SDK)
need to be installed for your operating system along with the video
driver. These packages include the CUDA compiler \cmd{nvcc}, which is
needed to develop executable code, and the graphics card driver that
allows your program to access the GPU card.

Download the relevant driver for your machine from:
\url{http://www.nvidia.com/Download/index.aspx?lang=en-us}
and the CUDA toolkit from:
\url{https://developer.nvidia.com/cuda-downloads}
Follow the instructions provided by \nvidia\ for the installation.


To ensure that all is installed correctly and working, you should
compile and run the SDK examples, which include many programs that
illustrate the capabilities of CUDA and the GPU; for example, \nvidia's
sorting program \cmd{radixSort} is used by GPUSPH to organize the
neighbor list. Some interesting SDK programs are \cmd{fluidsGL} and
\cmd{particles}. To compile the SDK programs, after the SDK is
installed, go to \url{/Developer/GPU Computing/C} and (on a unix/linux
or mac machine), type \cmd{make} on a terminal window command line. This
should create a directory of executable examples located within the C
directory called bin/darwin/release for the mac and bin/linux/release
for a linux machine. In this directory, type \cmd{./fluidsGL} to run
the \cmd{fluidsGL} example. You should see a green window open on your
desktop. Use the mouse to stir up the fluid. The example program
Particles is worth playing with as well, as it provided a basis for
developing GPUSPH.



\subsection{Installing GPUSPH}

The GPUSPH source code is hosted on \href{http://github.com}{GitHub}.
The project's GitHub page is \url{http://github.com/GPUSPH/gpusph}.

To obtain the GPUSPH code, you can either use the \cmd{git} revision
control system, or download a \cmd{.zip}ped archive of a specific
version. This manual refers to \currentver\ of GPUSPH.

If you have \cmd{git} installed, you can use
\begin{shellcode}[escapeinside=\{\}]
git clone https://github.com/GPUSPH/gpusph.git
cd gpusph
git checkout v{\version}
\end{shellcode}
to get \currentver\ specifically. Otherwise, download the \cmd{.zip}ped
archive from \url{http://github.com/GPUSPH/gpusph/archive/v\version.zip},
and then
\begin{shellcode}[escapeinside=\{\}]
unzip v{\version}.zip
cd gpusph-{\version}
\end{shellcode}
(you may remove \cmd{v}\version\cmd{.zip} afterwards).

Within the top directory, you can find the \cmd{Makefile}, a \cmd{src}
directory (holding the main GPUSPH source), a \cmd{scripts} directory
(holding various auxiliary scripts), a copy of the license, settings to
produce internal documentation with Doxygen, and a sample Digital
Elevation Model (DEM) data file.

The most interesting source files in \cmd{src} are the \cmd{Problem}s.
A few sample problems are shipped with GPUSPH, showing how to employ
specific features. You can get a list of the available problems by
running
\begin{shellcode}
make list-problems
\end{shellcode}

To build and test GPUSPH, you can run
\begin{shellcode}
make test
\end{shellcode}
which should automatically detect your configuration, such as the
compute capability of your GPU as well as the availability of optional
libraries such as MPI (for mulit-node support) or HDF5 (to read HDF5SPH
data files).

When the building completes, you will have some new directoryes
(\cmd{build} and \cmd{dist}) and a \cmd{GPUSPH} soft link to the
compiled binary. \cmd{make test} will also automatically run
\cmd{./GPUSPH} for you.

After building, simply runnning \cmd{./GPUSPH} will run the program
again.

\subsection{Installing the CHRONO library}

The CHRONO website provides information for how to install CHRONO:
\url{http://api.chrono.projectchrono.org/tutorial_install_chrono.html}

\textbf{Remark:} There is no need for the Irrlicht library with GPUSPH.

In this section we summarize the steps for the CHRONO library installation.
To install CHRONO, besides the GPUSPH requirements you need to have \cmd{cmake} installed
and a cmake interface like \cmd{ccmake} on Linux.

First, create a directory where to install CHRONO:
\begin{shellcode}
mkdir install_chrono
\end{shellcode}
In that directory, clone the CHRONO repository from Github in a source directory:
\begin{shellcode}
cd install_chrono
git clone https://github.com/projectchrono/chrono.git source
\end{shellcode}
This command will download the CHRONO repository in a folder named \cmd{source}.

Create a folder where to build CHRONO:
\begin{shellcode}
mkdir build
\end{shellcode}
From the \cmd{build} repository, run \cmd{cmake}:
\begin{shellcode}
cmake ../source
\end{shellcode}
Configure the compilation options with \cmd{ccmake}:
\begin{shellcode}
ccmake .
\end{shellcode}
set the following options to off:
\begin{shellcode}
 ENABLE_MODULE_CASCADE            OFF
 ENABLE_MODULE_COSIMULATION       OFF
 ENABLE_MODULE_FEA                OFF
 ENABLE_MODULE_FSI                OFF
 ENABLE_MODULE_IRRLICHT           OFF
 ENABLE_MODULE_MATLAB             OFF
 ENABLE_MODULE_MKL                OFF
 ENABLE_MODULE_OPENGL             OFF
 ENABLE_MODULE_PARALLEL           OFF
 ENABLE_MODULE_POSTPROCESS        OFF
 ENABLE_MODULE_PYTHON             OFF
 ENABLE_MODULE_VEHICLE            OFF
 ENABLE_OPENMP                    OFF
\end{shellcode}

Once this is done, you can compile the CHRONO project
(still from the \cmd{build} folder):
\begin{shellcode}
make
\end{shellcode}
and install the library (also from the \cmd{build} folder):
\begin{shellcode}
sudo make install
\end{shellcode}

%\todo{How to install relevant packages on common distributions
%(Debian/Ubuntu, Arch, Fedora/RedHat).}


\section{Choosing the GPUSPH \cmd{Problem} and other compilation options}\label{sec:compileoptions}

You can test a different problem by using:
\begin{shellcode}
make OtherProblem test
\end{shellcode}
where \cmd{OtherProblem} is the name of a different problem. You can get
a list of available problems with \cmd{make list-problems}.

There are a number of other options available. A complete list of the
options and their description can be obtained by running \cmd{make
help-options}. All options (with the exception of \cmd{plain} and
\cmd{echo}) are persistent across compilations, so they can be set once
with \cmd{make option=value}, and subsequent executions of \cmd{make}
will remember the \cmd{value} set.

%\todo{Describe \cmd{make} options}
The \cmd{make} options are listed below:
\begin{itemize}
\item \cmd{target_arch} - if set to 32, force compilation for 32 bit architecture
\item \cmd{problem} - Name of the problem. Since version 5, you can just use the problem name as a target to build that particular test case.
\item \cmd{dbg} - 0 no debugging, 1 enable debugging
\item \cmd{compute} - 11, 12, 13, 20, 21, 30, 35, etc: compute capability to compile for (default: autodetect)
\item \cmd{fastmath} - Enable or disable fastmath. Default: 0 (disabled)
\item \cmd{mpi} - 0 do not use MPI (no multi-node support), 1 use MPI (enable multi-node support). Default: autodetect
\item \cmd{hdf5} - 0 do not use HDF5, 1 use HDF5, 2 use HDF5 and HDF5 requires MPI. Default: autodetect
\item \cmd{verbose} - 0 quiet compiler, 1 ptx assembler, 2 all warnings
\item \cmd{plain} - 0 fancy line-recycling stage announce, 1 plain multi-line stage announce
\item \cmd{echo} - 0 silent, 1 show commands
\item \cmd{chrono} - 0 do not use the CHRONO library, 1 use the CHRONO library
\end{itemize}
To view your current make options type \cmd{make show} instead of make.

\section{Example Problems}

Simulations in GPUSPH are defined in terms of \cmd{Problem}s. Some
example problems are provided with GPUSPH itself, to illustrate the
basics of problem design, and how to use the fundamental building blocks
provided by GPUSPH. Such building blocks include a variety of
geometrical shapes to describe the (fixed) solid boundaries of the
domain, as well as a number of objects that move following prescribed
laws, such as gates, pistons and paddles.

These objects are designed to offer great flexibility in their use, far
beyond what is shown in the sample problems. This flexibility should
allow you to create very complex simulations by combining the objects
appropriately.

The number of particles used in the test problems is deliberately taken
as a small number, simply to allow for fast execution times even on
older hardware. One of the first tests to try is to increase the
resolution by reducing the size of the particles. For example,
by~reducing the particle size from the default of~$0.025$m to the
smalle~$0.02$m, \cmd{DamBreak3D} would run with $21,252$ particles
instead of the default $10,664$.

This can be done in two ways. A permanent change comes about by editing
the problem file (e.g. \cmd{DamBreak3D.cc}) and changing the value
passed as argument of \cmd{set_deltap()} (e.g., replace
\cmd{set_deltap(0.025f);} with \cmd{set_deltap(0.02f);}. The second way
is to specify the particle size at runtime using the appropriate command
line option (described below): e.g. \cmd{./GPUSPH --deltap 0.02}.

\subsection{DamBreak3D}

\cmd{DamBreak3D} is a case originally used by
\cite{gomez-gesteira_using_2004}
for testing a prototype version of SPHysics. It is based on some
experiments done by \cite{arnason_interactions_2005} at the University of Washington.
We assume an instantaneous breaking dam and the resulting flow impinging
onto a rectangular object. The whole problem is contained within a
bounding box, which extends $1.6$m in length ($x$ axis), $0.67$m in
width ($y$ axis), and $0.4$m in height. This is the experimental box.
The fluid behind the dam is a rectangular box of water at one end of the
tank at time equal to zero. The dam is assumed to break instantaneously
so that the column of water, confined on three sides, collapses into the
tank. In the tank there is a vertical rectangular object -- the
collapsing water column impacts on the tank and then flows up the front
face of the object and around the sides. Finally the water hits the back
wall of the tank. A screenshot of the simulation at time $0.6s$ is provided
in the Figure \ref{fig:DamBreak3D}.

\begin{figure}[h]
  \begin{center}
    \includegraphics[scale=0.3, trim={100 100 100 100},clip]{../fig/damBreak3D_screenshot.png}
    \caption{Screenshot of the DamBreak3D simulation at time $0.6s$.}\label{fig:DamBreak3D}   
  \end{center}
\end{figure}

\subsection{DamBreakGate}

In most laboratory experiments of dam breaks, the dam takes a certain
amount of time to move out of the way. The example problem
\cmd{DamBreakGate} illustrates the use of moving boundaries
with prescribed motions. The problem is set up the same way as the
\cmd{DamBreak3D} case, but there is a moving gate that is raised
vertically with a linearly varying velocity. In this case, the gate will
move with a velocity that is zero when the problem starts and that
linearly increases with time until the gate is outside the domain. The
effect on the dam break is that the escaping water is affected by the
gage motion. (See \cite{crespo_modeling_2008}'s SPH modeling of
\cite{janosi_turbulent_2004}'s experiment, where a moving gate was important.)

The moving gate is created by defining its geometry with a geometry type
\cmd{GT_MOVING_BODY}, and overriding the Problem
\cmd{moving_bodies_callback} function to determine the linear and
angular velocity of the body.

A screenshot of the simulation at time $0.8s$ is provided
in the Figure \ref{fig:DamBreakGate}.

\begin{figure}[h]
  \begin{center}
    \includegraphics[scale=0.35, trim={100 100 100 100},clip]{../fig/damBreakGate_screenshot.png}
    \caption{Screenshot of the DamBreakGate simulation at time $0.8s$.}\label{fig:DamBreakGate}   
  \end{center}
\end{figure}

\subsection{OpenChannel}

This problem represents an instantaneous start up of a highly viscous
end dense fluid flow in an open channel on a $9\deg$ slope. The
channel is rectangular in cross-section ($1$m wide and $0.7$m deep) and
the computed length of the infinitely long channel is $2$m. The side
walls are fixed (Leonard-Jones boundary force) while the computational
ends of the domain are periodic, so that a particle leaving the
downstream end of the model domain enters the upstream end at the same
place, $2$m upstream.

The periodic boundary here is used in the $x$ direction, although
boundaries in other problems can be periodic in the other directions as
well. The key parameter in the problem statement is the simulation
framework \cmd{periodicity}, which can be set to any combination of
\cmd{PERIODIC_X}, \cmd{PERIODIC_Y}, \cmd{PERIODIC_Z} to indicate
peridocity along each of the axes.
Figure \ref{fig:OpenChannel} shows the shape of the velocity field in 
the channel after time convergence.
\begin{figure}[h]
  \begin{center}
    \includegraphics[scale=0.35, trim={0 100 0 0},clip]{../fig/openChannel_screenshot.png}
    \caption{Screenshot of the OpenChannel simulation after time convergence.}\label{fig:OpenChannel}
  \end{center}
\end{figure}

\subsection{WaveTank}

WaveTank uses a moving boundary to create a paddle wavemaker at one end
of a wave tank with a sloping bottom (bottom slope is $4.2364\deg$). The
wavemaker motion is controlled by the \cmd{moving_bodies_callback} function. In
this case, the length of the paddle is $1.0$m and the paddle pivots
about an origin \cmd{m_origin}; here, the pivot is located $0.1344$m
below the bottom and $0.13$m from the front wall of the tank. To specify
the paddle motion, the angular frequency of the motion ($2 \pi/T$, where
$T=1$s is the wave period), and the wave paddle stroke at the water
surface ($S=0.1$m) are given in the variables \cmd{mb_omega} and
\cmd{mb_amplitude}. To change the stroke and the frequency of the wave
paddle, you must change these variables in the problem file,
\cmd{WaveTank.cc}.
Figure \ref{fig:WaveTank} shows a screenshot of the simulation at time $9s$.

\iffalse
\begin{figure}[h]
\centering{%
\includegraphics[width=0.63\textwidth]{paddle.png}%
}
\caption{Schematic of the wave paddle for \cmd{WaveTank.cc}}
\end{figure}
\else
%\todo{paddle picture}
\fi

\begin{figure}[h]
  \begin{center}
    \includegraphics[scale=0.35, trim={0 200 0 200},clip]{../fig/waveTank_screenshot.png}
    \caption{Screenshot of the WaveTank simulation at time $9s$.}\label{fig:WaveTank}
  \end{center}
\end{figure}

\subsection{SolitaryWave}

SolitaryWave is similar in set up to the WaveTank example, except that a
piston moving boundary is used. The motion of a vertical plate is
determined by the method of \cite{goring_tsunamis_1979}, available in PDF format
from:
\cmd{http://caltechkhr.library.caltech.edu/50/}
The full excursion (stroke) of the paddle is the variable \cmd{S}.

\subsection{Seiche}

The Seiche problem is to examine the influence of shaking on a
rectangular container of size: $\ell = 0.707$m, $w = \ell/2$, and depth,
$H = 0.5$m. The purpose of the example is to illustrate the ability to
vary gravity in a problem. As the problem starts, there is water in the
container. After $0.3$s, gravity is modified by adding a component in
the $x$ direction, such that the total gravity vector is
\cmd{set_gravity(3.*sin(9.8*(t-m_gtstart)), 0.0, -9.81f);}
which means that the container is shaken with a sinusoidal
motion with angular frequency of $9.8s^{-1}$ (period${} = 0.64$s),
with a magnitude of $3\text{m}/\text{s}^2$ until time \cmd{m_gtend=3.0}
is reached, when the gravity vector once again returns to the vertical
acceleration of gravity. After this time, the seiching motion starts to
decrease in amplitude.

\iffalse
\begin{figure}[h]
\centering{%
\includegraphics[scale=0.5]{Seiche.png}%
}
\caption{Resonant seiching in a rectangular domain showing the results
of a time varying gravity in the problem, \cmd{Seiche.cc}. Here the tank has
been shaking side to side at the resonant frequency of $0.638$s. The
color coding is for the pressure in the fluid.}
\end{figure}
\else
%\todo{seiche picture}
\fi

The variation of gravity with time (and any stop (\cmd{m_gtend}) and
start times) is prescribed in a user-supplied (in the problem)
\cmd{g_callback} function.

\subsection{DEMExample}

This is an example showing how to use GPUSPH's support for Digital
Elevation Models (DEMs). It loads the topography of the bottom of the
domain from a file called \cmd{half_wave0.1m.txt}, shipped with GPUSPH.
A different DEM can be used, by either changing the name in the source
\cmd{DEMExample.cu} file, or by providing the new name as argument to the
\cmd{--dem} command-line option to GPUSPH.

%\todo{TestTopo picture}

\section{GPUSPH Command Line Options}\label{options}

When running from the command line, there are several options available
to you to alter some aspects of the GPUSPH run.

\begin{description}
\item [-{}-device \emph{integer}]
Choose which GPU(s) to use for the run.
On the command line: \cmd{./GPUSPH --device N}, where N is the
(integer) number of the device you wish to use. 
To find the number associated with each of your CUDA-enabled 
devices (graphics cards), you
can use the CUDA SDK program DeviceQueryDrv. If you only have one
CUDA-enabled GPU, the only possible choice for N is~0, which is the
default.
If you want to run the simulation on several GPUs, the command is:
\cmd{./GPUSPH --device i,j,k}
\item [-{}-deltap \emph{float}]
Change the resolution (inter-particle spacing) at which the problem
should be run.
\item [-{}-tend \emph{float}]
The model time in seconds when you wish the model to stop.
\item [-{}-dt \emph{float}]
Force the use of the fixed, specified time-step, even if the problem
would use dynamic time-stepping otherwise.
\item [-{}-dem \emph{string}]
For the Problem DEMExample: the name of the DEM file to use.
\item [-{}-resume \emph{fname}]
Resume from the given file (HotStart file saved by HotWriter).
\item [-{}-checkpoint-every \emph{float}]
HotStart checkpoints will be created every VAL seconds of simulated time (float VAL, 0 disables).
\item [-{}-checkpoints \emph{integer}]
Number of HotStart checkpoints to keep.
\item [-{}-maxiter \emph{integer}]
Break after this many iterations.
\item [-{}-dir \emph{string}]
Use given directory for dumps instead of date-based one.
\item [-{}-nosave]
Disable all file dumps but the last.
\item [-{}-gpudirect]
Enable GPUDirect for RDMA (requires a CUDA-aware MPI library).
\item [-{}-striping]
Enable computation/transfer overlap in multi-GPU (usually convenient for 3+ devices).
\item [-{}-asyncmpi]
Enable asynchronous network transfers (requires GPUDirect and 1 process per device).
\item [-{}-num-hosts \emph{integer}]
Uses multiple processes per node by specifying the number of nodes.
\item [-{}-byslot-scheduling]
MPI scheduler is filling hosts first, as opposite to round robin scheduling.
\item [-{}-debug \emph{flags}]
Enable specified debug flags.
\item [-{}-help]
Show the help and exit.
\end{description}

\subsection{Debugging GPUSPH execution}

To assist developers when introducing new GPUSPH features, it is
possible to trace the execution of the simulator, providing insight on
the evolution of some of the internal structures using the provided
debug flags. Multiple debug flags can be enabled in a single run, by
passing a comma-separated list as an argument to \cmd{--debug}. For
example:
\begin{shellcode}
./GPUSPH --debug print_step,inspect_buffer_access
\end{shellcode}

The \cmd{validate_init_position} debug option is also useful for users
that want to further validate their initial setup.

The debug flags currently supported are:
\begin{description}
\item[print_step] print each step and command as it is being executed;
\item[neibs] debug the neighbors list on host, saving it to file;
\item[forces] debug forces on host, saving them to file;
\item[numerical_density] debug relative density variation on host, saving it to
file;
\item[inspect_preforce] inspect pre-force particle status; this saves
the particle system status before each force computation;
\item[inspect_pregamma] inspect pre-gamma integration particle status;
this saves the particle system status before the gamma integration
kernel execution;
\item[inspect_buffer_access] inspect buffer access; show how each buffer
is being accessed (reading or writing, and from which state) during
execution of each command; note that this needs compile-time support,
by enabling \cmd{#define DEBUG_BUFFER_ACCESS 1} in \cmd{src/buffer.h};
\item[inspect_buffer_lists] inspect the state of the particle system and
its buffer lists (states and free buffer pool) after each command;
\item[check_buffer_update] check buffer update; this is used to check
that all and only modified buffers are exchanged in an
\cmd{UPDATE_EXTERNAL} call;
\item[check_buffer_consistency] check buffer consistency;
when this is enabled, after every command GPUSPH will verify
that the shared parts of the subdomains in multi-GPU simulations
are consistent across devices; note this needs compile-time support,
by adding \cmd{-DINSPECT_DEVICE_MEMORY} to \cmd{CPPFLAGS} in
\cmd{Makefile.local};
\item[clobber_invalid_buffers] clobber invalid buffers;
when this is true, every time a buffer is marked invalid,
its content will be clobbered (reset to the initial value,
typically NAN or equivalent); useful to check that stale data is not
being used inadvertently;. note that this needs compile-time support,
by enabling \cmd{#define DEBUG_BUFFER_ACCESS 1} in \cmd{src/buffer.h};
\item[validate_init_positions] throw an exception (instead of just
warning) if a particle is out of bounds during initialization of the
problem;
\item[benchmark_command_runtimes] measure (and show) command runtimes.
\end{description}

\section{Running multi-node simulations}

GPUSPH can distribute the computation of a simulation on multiple 
GPU devices attached to different nodes of a cluster in different ways.

Say we want to launch a simulation on $N$ nodes, each with $D$ devices 
(with CUDA device numbers ranging from $0$ to $D-1$); 
the total number of devices in the simulation will be $N \times D$. We can run either:
\begin{itemize}
\item one process per node, $D$ GPUs per process
\item $2$ processes per node, $D/2$ GPUs per process
\item $4$ processes per node, $D/4$ GPUs per process
\item etc.
\end{itemize}

Additionally, some MPI implementations have built-in support for CUDA, 
which allows for faster communication between devices on different nodes. 
Experimental support for this feature can be enabled in GPUSPH with the \cmd{--gpudirect} command-line option.

The best decision on how to distribute the computation across nodes and devices depends on 
the queue policy of the cluster, on the network topology, on simple a posteriori performance tests, 
on the capabilities of the MPI implementation, etc.

If we wanted to run the simulation on all the devices of one node, we would run in an interactive shell:

\begin{shellcode}
./GPUSPH --device 0,1,...D-1
\end{shellcode}

Running the same simulation on multiple nodes only requires to run 
the same command within the reference MPI launcher (usually a script called \cmd{mpirun}); 
GPUSPH will retrieve the necessary information about the launch environment 
directly from the MPI runtime and will organize the node-to-node communication accordingly.

\begin{shellcode}
mpirun -np N ./GPUSPH --device 0,1,...D-1
\end{shellcode}

This command leaves to MPI the choice of which nodes to use in the network, 
if more than N are available. 
It is always safe to provide MPI a list of hostnames corresponding 
to the nodes chosen to run the simulation. 
If the file containing the list of hostnames is called \cmd{myhostsfile}, a typical syntax will be:

\begin{shellcode}
mpirun -np N -hostfile ./myhostsfile \
./GPUSPH --device 0,1,...D-1
\end{shellcode}

Please note the syntax may vary from one MPI library to another. 
For example, MVAPICH uses \cmd{-hostfile} while OpenMPI \cmd{--hostfile}.

Let us now see the command line options needed to run the same simulation 
with more processes (and thus on more nodes) and a smaller number of devices per process. 
In this case we need to inform both the MPI runtime and GPUSPH. 
For the former, we simply decrease the number of processes to start (with the \cmd{-np} option); 
for the latter, we need to shorten appropriately the list of devices 
passed with the \cmd{--device} option. 
If our aim is to run $N*2$ processes each using $D/2$ devices, we then run:
\begin{shellcode}
mpirun -np N*2 -hostfile ./myhostsfile \
./GPUSPH --device 0,1,...D/2-1
\end{shellcode}

Here we need to take care of a few important details. If the list of available 
hosts contains at least $N*2$ hostnames, the MPI runtime will start every process 
on a different physical node. But what happens if the list is shorter 
(e.g. $N$ hosts only), or if we want anyway to use a smaller number of 
nodes (for example because part of the cluster is already used by other processes)? 
The MPI runtime will start multiple processes per node and the GPU device 
numbers will be likely to conflict (i.e. two different processes might 
try to use the same GPU device for different parts of the simulation domain, 
causing a performance slowdown or a failure if the CUDA devices are set
in ``exclusive mode''). 
In this case, we must inform GPUSPH that the number of physical hosts (i.e. nodes) 
is smaller than the number of processes, so that it will shift the GPU device 
numbers of the appropriate processes and no GPU device will be accessed by two processes. 
The corresponding option is \cmd{--num-hosts}:

\begin{shellcode}
mpirun -np N*2 -hostfile ./myhostsfile \
--num-hosts N ./GPUSPH --device 0,1,...D/2-1
\end{shellcode}

But there is another important detail. There are different ways the MPI 
runtime can distribute the processes across the nodes. 
Two very common policies are ``by slot'' (fill-first) and ``by node''
(round robin). 
The scheduling policy affects the association between the process ranks 
and the CUDA device numbers, so GPUSPH must be informed about it to use 
the appropriate offsets. GPUSPH assumes a round robin schedule is being used; 
if this is not true, the \cmd{--byslot-scheduling} option must be passed:

\begin{shellcode}
mpirun -np N*2 -hostfile ./myhostsfile \
--num-hosts N --byslot-scheduling \
./GPUSPH --device 0,1,...D/2-1
\end{shellcode}

There is no optimal policy in general as its performance depends on the node 
load and the node-to-node connection speed. It is worth trying both to check 
whether one is more performant than the other. The default policy is usually 
``by slot''(fill-first, usually preferred by non GPU-based softwares) but it
is always safer to explicitly set it for every run. 
In OpenMPI the corresponding options are \cmd{--byslot} and \cmd{--bynode}.

One final possibility is to run one process per device. This is the most 
consuming option from the point of view of the host memory, since every process 
will allocate on host the whole simulation scenario, but it might be useful if the 
GPUDirect feature is being used but the MPI runtime does not support multiple 
devices per process, or if the amount of data to be saved on file is large or 
the saving frequency is very high, so that saving would benefit from a parallel 
dump (every process saves its part of the simulation independently from the other processes).

Finally, let's see some practical examples. Suppose our cluster has 12 nodes, each equipped
with 4 GPU devices. We need to run a simulation on 12 devices. 
To run 3 processes on 3 hosts, each using 4 devices, we will run:

\begin{shellcode}
mpirun -np 3 -hostfile ./myhostsfile ./GPUSPH --device 0,1,2,3
\end{shellcode}

To run 6 processes on 6 hosts, each using 2 devices, we will run:

\begin{shellcode}
mpirun -np 6 -hostfile ./myhostsfile ./GPUSPH --device 0,1
\end{shellcode}

Note that another simulation can be run simultaneously on the remaining 2 devices of the same nodes, with:
\begin{shellcode}
mpirun -np 6 -hostfile ./myhostsfile ./GPUSPH --device 2,3
\end{shellcode}

To run 6 processes on 3 hosts, each using 2 devices, we will run:

\begin{shellcode}
mpirun -np 6 -hostfile ./myhostsfile_with3hosts \
--num-hosts 3 ./GPUSPH --device 0,1
\end{shellcode}

(--byslot-scheduling might also be necessary)

To run 12 processes on 12 hosts, each using 1 device, we will run:

\begin{shellcode}
mpirun -np 12 -hostfile ./myhostsfile ./GPUSPH --device 0
\end{shellcode}

Note that another simulation can be run simultaneously on one of the free devices of each node (e.g. device number 3), with:

\begin{shellcode}
mpirun -np 12 -hostfile ./myhostsfile ./GPUSPH --device 3
\end{shellcode}

Please note the options for the MPI library always precede the GPUSPH executable name.
If the MPI library supports it, we also suggest enabling the option to tag each 
line of the output with the process rank that has generated it; 
in OpenMPI, the option is called \cmd{--tag-output} while in MVAPICH \cmd{-prepend-rank}. 
This will come very helpful when the logs need to be analyzed (tip: use grep to separate the logs if they are multiplexed).

If you need to use any queue-management system, remember to inform it about the desired topology, 
coherently with the options passed to GPUSPH. For example, with PBS you would set the \cmd{nodes} 
and \cmd{ppn} parameters for the number of hosts and processes per host, respectively.


\section{Installing pre/post processing tools}

\subsection{Installing SALOME}
You can download SALOME from:
\url{http://www.salome-platform.org/downloads/current-version}

For this you need to register on the SALOME website.
Then, follow the installation instructions from the SALOME website and the
installer.

\textbf{Remark}: SALOME is used for SA boundary pre-processing, to generate an
STL mesh of the boundaries. You can of course use another mesher of your 
choice for this step.

\subsection{Installing CRIXUS}
Crixus is a preprocessing tool for GPUSPH.

\textbf{Prerequisites}:
\begin{itemize}
\item cmake $\ge$ 2.8
\item cuda
\item hdf5 $\ge$ 1.8.7
\end{itemize}

\textbf{Getting CRIXUS}:
\begin{shellcode}
git clone https://github.com/Azrael3000/Crixus.git
cd Crixus.git
\end{shellcode}

\textbf{Compiling Crixus}:
Crixus uses CMake for compilation. 
Let us assume that you have CRIXUS in a \cmd{Crixus.git} directory. 
and you want the building to happen in \cmd{Crixus.git/build} then follow the commands below:
\begin{shellcode}
mkdir build
cd build
cmake ..
make
\end{shellcode}
Note that you should not run cmake in the main Crixus folder.

The binary is then located at \cmd{Crixus.git/build/bin/Release/Crixus}.
Note that \cmd{make install} is not supported yet. To easily change the parameters of cmake you can use ccmake instead.

If hdf5 cannot be found due to lacking environmental variable you can edit the main CMakeLists.txt which has a commented line that reads:
\begin{shellcode}
#set(ENV{HDF5_ROOT} "/your/path/to/hdf5")
\end{shellcode}
Uncomment it and set the respective hdf5 path in order to use your custom installation.

To finish the installation it is recommended to add the path to the CRIXUS binary to your
\cmd{PATH} environment variable. Add this line in \cmd{\~/.bashrc}:
\begin{shellcode}
export PATH=/your_path/Crixus.git/build/bin/Release/Crixus:$PATH
\end{shellcode}
where \cmd{/your_path} is your path to the CRIXUS directory.

\subsection{Installing PARAVIEW}

PARAVIEW is directly available from the Linux packages.

\newpage
\appendixpage
\appendix
% vi:tw=72:fenc=utf-8:ft=tex
\section{GNU General Public License}
\begin{verbatim}

                    GNU GENERAL PUBLIC LICENSE
                       Version 3, 29 June 2007

 Copyright (C) 2007 Free Software Foundation, Inc. <http://fsf.org/>
 Everyone is permitted to copy and distribute verbatim copies
 of this license document, but changing it is not allowed.

                            Preamble

  The GNU General Public License is a free, copyleft license for
software and other kinds of works.

  The licenses for most software and other practical works are designed
to take away your freedom to share and change the works.  By contrast,
the GNU General Public License is intended to guarantee your freedom to
share and change all versions of a program--to make sure it remains free
software for all its users.  We, the Free Software Foundation, use the
GNU General Public License for most of our software; it applies also to
any other work released this way by its authors.  You can apply it to
your programs, too.

  When we speak of free software, we are referring to freedom, not
price.  Our General Public Licenses are designed to make sure that you
have the freedom to distribute copies of free software (and charge for
them if you wish), that you receive source code or can get it if you
want it, that you can change the software or use pieces of it in new
free programs, and that you know you can do these things.

  To protect your rights, we need to prevent others from denying you
these rights or asking you to surrender the rights.  Therefore, you have
certain responsibilities if you distribute copies of the software, or if
you modify it: responsibilities to respect the freedom of others.

  For example, if you distribute copies of such a program, whether
gratis or for a fee, you must pass on to the recipients the same
freedoms that you received.  You must make sure that they, too, receive
or can get the source code.  And you must show them these terms so they
know their rights.

  Developers that use the GNU GPL protect your rights with two steps:
(1) assert copyright on the software, and (2) offer you this License
giving you legal permission to copy, distribute and/or modify it.

  For the developers' and authors' protection, the GPL clearly explains
that there is no warranty for this free software.  For both users' and
authors' sake, the GPL requires that modified versions be marked as
changed, so that their problems will not be attributed erroneously to
authors of previous versions.

  Some devices are designed to deny users access to install or run
modified versions of the software inside them, although the manufacturer
can do so.  This is fundamentally incompatible with the aim of
protecting users' freedom to change the software.  The systematic
pattern of such abuse occurs in the area of products for individuals to
use, which is precisely where it is most unacceptable.  Therefore, we
have designed this version of the GPL to prohibit the practice for those
products.  If such problems arise substantially in other domains, we
stand ready to extend this provision to those domains in future versions
of the GPL, as needed to protect the freedom of users.

  Finally, every program is threatened constantly by software patents.
States should not allow patents to restrict development and use of
software on general-purpose computers, but in those that do, we wish to
avoid the special danger that patents applied to a free program could
make it effectively proprietary.  To prevent this, the GPL assures that
patents cannot be used to render the program non-free.

  The precise terms and conditions for copying, distribution and
modification follow.

                       TERMS AND CONDITIONS

  0. Definitions.

  "This License" refers to version 3 of the GNU General Public License.

  "Copyright" also means copyright-like laws that apply to other kinds of
works, such as semiconductor masks.

  "The Program" refers to any copyrightable work licensed under this
License.  Each licensee is addressed as "you".  "Licensees" and
"recipients" may be individuals or organizations.

  To "modify" a work means to copy from or adapt all or part of the work
in a fashion requiring copyright permission, other than the making of an
exact copy.  The resulting work is called a "modified version" of the
earlier work or a work "based on" the earlier work.

  A "covered work" means either the unmodified Program or a work based
on the Program.

  To "propagate" a work means to do anything with it that, without
permission, would make you directly or secondarily liable for
infringement under applicable copyright law, except executing it on a
computer or modifying a private copy.  Propagation includes copying,
distribution (with or without modification), making available to the
public, and in some countries other activities as well.

  To "convey" a work means any kind of propagation that enables other
parties to make or receive copies.  Mere interaction with a user through
a computer network, with no transfer of a copy, is not conveying.

  An interactive user interface displays "Appropriate Legal Notices"
to the extent that it includes a convenient and prominently visible
feature that (1) displays an appropriate copyright notice, and (2)
tells the user that there is no warranty for the work (except to the
extent that warranties are provided), that licensees may convey the
work under this License, and how to view a copy of this License.  If
the interface presents a list of user commands or options, such as a
menu, a prominent item in the list meets this criterion.

  1. Source Code.

  The "source code" for a work means the preferred form of the work
for making modifications to it.  "Object code" means any non-source
form of a work.

  A "Standard Interface" means an interface that either is an official
standard defined by a recognized standards body, or, in the case of
interfaces specified for a particular programming language, one that
is widely used among developers working in that language.

  The "System Libraries" of an executable work include anything, other
than the work as a whole, that (a) is included in the normal form of
packaging a Major Component, but which is not part of that Major
Component, and (b) serves only to enable use of the work with that
Major Component, or to implement a Standard Interface for which an
implementation is available to the public in source code form.  A
"Major Component", in this context, means a major essential component
(kernel, window system, and so on) of the specific operating system
(if any) on which the executable work runs, or a compiler used to
produce the work, or an object code interpreter used to run it.

  The "Corresponding Source" for a work in object code form means all
the source code needed to generate, install, and (for an executable
work) run the object code and to modify the work, including scripts to
control those activities.  However, it does not include the work's
System Libraries, or general-purpose tools or generally available free
programs which are used unmodified in performing those activities but
which are not part of the work.  For example, Corresponding Source
includes interface definition files associated with source files for
the work, and the source code for shared libraries and dynamically
linked subprograms that the work is specifically designed to require,
such as by intimate data communication or control flow between those
subprograms and other parts of the work.

  The Corresponding Source need not include anything that users
can regenerate automatically from other parts of the Corresponding
Source.

  The Corresponding Source for a work in source code form is that
same work.

  2. Basic Permissions.

  All rights granted under this License are granted for the term of
copyright on the Program, and are irrevocable provided the stated
conditions are met.  This License explicitly affirms your unlimited
permission to run the unmodified Program.  The output from running a
covered work is covered by this License only if the output, given its
content, constitutes a covered work.  This License acknowledges your
rights of fair use or other equivalent, as provided by copyright law.

  You may make, run and propagate covered works that you do not
convey, without conditions so long as your license otherwise remains
in force.  You may convey covered works to others for the sole purpose
of having them make modifications exclusively for you, or provide you
with facilities for running those works, provided that you comply with
the terms of this License in conveying all material for which you do
not control copyright.  Those thus making or running the covered works
for you must do so exclusively on your behalf, under your direction
and control, on terms that prohibit them from making any copies of
your copyrighted material outside their relationship with you.

  Conveying under any other circumstances is permitted solely under
the conditions stated below.  Sublicensing is not allowed; section 10
makes it unnecessary.

  3. Protecting Users' Legal Rights From Anti-Circumvention Law.

  No covered work shall be deemed part of an effective technological
measure under any applicable law fulfilling obligations under article
11 of the WIPO copyright treaty adopted on 20 December 1996, or
similar laws prohibiting or restricting circumvention of such
measures.

  When you convey a covered work, you waive any legal power to forbid
circumvention of technological measures to the extent such circumvention
is effected by exercising rights under this License with respect to
the covered work, and you disclaim any intention to limit operation or
modification of the work as a means of enforcing, against the work's
users, your or third parties' legal rights to forbid circumvention of
technological measures.

  4. Conveying Verbatim Copies.

  You may convey verbatim copies of the Program's source code as you
receive it, in any medium, provided that you conspicuously and
appropriately publish on each copy an appropriate copyright notice;
keep intact all notices stating that this License and any
non-permissive terms added in accord with section 7 apply to the code;
keep intact all notices of the absence of any warranty; and give all
recipients a copy of this License along with the Program.

  You may charge any price or no price for each copy that you convey,
and you may offer support or warranty protection for a fee.

  5. Conveying Modified Source Versions.

  You may convey a work based on the Program, or the modifications to
produce it from the Program, in the form of source code under the
terms of section 4, provided that you also meet all of these conditions:

    a) The work must carry prominent notices stating that you modified
    it, and giving a relevant date.

    b) The work must carry prominent notices stating that it is
    released under this License and any conditions added under section
    7.  This requirement modifies the requirement in section 4 to
    "keep intact all notices".

    c) You must license the entire work, as a whole, under this
    License to anyone who comes into possession of a copy.  This
    License will therefore apply, along with any applicable section 7
    additional terms, to the whole of the work, and all its parts,
    regardless of how they are packaged.  This License gives no
    permission to license the work in any other way, but it does not
    invalidate such permission if you have separately received it.

    d) If the work has interactive user interfaces, each must display
    Appropriate Legal Notices; however, if the Program has interactive
    interfaces that do not display Appropriate Legal Notices, your
    work need not make them do so.

  A compilation of a covered work with other separate and independent
works, which are not by their nature extensions of the covered work,
and which are not combined with it such as to form a larger program,
in or on a volume of a storage or distribution medium, is called an
"aggregate" if the compilation and its resulting copyright are not
used to limit the access or legal rights of the compilation's users
beyond what the individual works permit.  Inclusion of a covered work
in an aggregate does not cause this License to apply to the other
parts of the aggregate.

  6. Conveying Non-Source Forms.

  You may convey a covered work in object code form under the terms
of sections 4 and 5, provided that you also convey the
machine-readable Corresponding Source under the terms of this License,
in one of these ways:

    a) Convey the object code in, or embodied in, a physical product
    (including a physical distribution medium), accompanied by the
    Corresponding Source fixed on a durable physical medium
    customarily used for software interchange.

    b) Convey the object code in, or embodied in, a physical product
    (including a physical distribution medium), accompanied by a
    written offer, valid for at least three years and valid for as
    long as you offer spare parts or customer support for that product
    model, to give anyone who possesses the object code either (1) a
    copy of the Corresponding Source for all the software in the
    product that is covered by this License, on a durable physical
    medium customarily used for software interchange, for a price no
    more than your reasonable cost of physically performing this
    conveying of source, or (2) access to copy the
    Corresponding Source from a network server at no charge.

    c) Convey individual copies of the object code with a copy of the
    written offer to provide the Corresponding Source.  This
    alternative is allowed only occasionally and noncommercially, and
    only if you received the object code with such an offer, in accord
    with subsection 6b.

    d) Convey the object code by offering access from a designated
    place (gratis or for a charge), and offer equivalent access to the
    Corresponding Source in the same way through the same place at no
    further charge.  You need not require recipients to copy the
    Corresponding Source along with the object code.  If the place to
    copy the object code is a network server, the Corresponding Source
    may be on a different server (operated by you or a third party)
    that supports equivalent copying facilities, provided you maintain
    clear directions next to the object code saying where to find the
    Corresponding Source.  Regardless of what server hosts the
    Corresponding Source, you remain obligated to ensure that it is
    available for as long as needed to satisfy these requirements.

    e) Convey the object code using peer-to-peer transmission, provided
    you inform other peers where the object code and Corresponding
    Source of the work are being offered to the general public at no
    charge under subsection 6d.

  A separable portion of the object code, whose source code is excluded
from the Corresponding Source as a System Library, need not be
included in conveying the object code work.

  A "User Product" is either (1) a "consumer product", which means any
tangible personal property which is normally used for personal, family,
or household purposes, or (2) anything designed or sold for incorporation
into a dwelling.  In determining whether a product is a consumer product,
doubtful cases shall be resolved in favor of coverage.  For a particular
product received by a particular user, "normally used" refers to a
typical or common use of that class of product, regardless of the status
of the particular user or of the way in which the particular user
actually uses, or expects or is expected to use, the product.  A product
is a consumer product regardless of whether the product has substantial
commercial, industrial or non-consumer uses, unless such uses represent
the only significant mode of use of the product.

  "Installation Information" for a User Product means any methods,
procedures, authorization keys, or other information required to install
and execute modified versions of a covered work in that User Product from
a modified version of its Corresponding Source.  The information must
suffice to ensure that the continued functioning of the modified object
code is in no case prevented or interfered with solely because
modification has been made.

  If you convey an object code work under this section in, or with, or
specifically for use in, a User Product, and the conveying occurs as
part of a transaction in which the right of possession and use of the
User Product is transferred to the recipient in perpetuity or for a
fixed term (regardless of how the transaction is characterized), the
Corresponding Source conveyed under this section must be accompanied
by the Installation Information.  But this requirement does not apply
if neither you nor any third party retains the ability to install
modified object code on the User Product (for example, the work has
been installed in ROM).

  The requirement to provide Installation Information does not include a
requirement to continue to provide support service, warranty, or updates
for a work that has been modified or installed by the recipient, or for
the User Product in which it has been modified or installed.  Access to a
network may be denied when the modification itself materially and
adversely affects the operation of the network or violates the rules and
protocols for communication across the network.

  Corresponding Source conveyed, and Installation Information provided,
in accord with this section must be in a format that is publicly
documented (and with an implementation available to the public in
source code form), and must require no special password or key for
unpacking, reading or copying.

  7. Additional Terms.

  "Additional permissions" are terms that supplement the terms of this
License by making exceptions from one or more of its conditions.
Additional permissions that are applicable to the entire Program shall
be treated as though they were included in this License, to the extent
that they are valid under applicable law.  If additional permissions
apply only to part of the Program, that part may be used separately
under those permissions, but the entire Program remains governed by
this License without regard to the additional permissions.

  When you convey a copy of a covered work, you may at your option
remove any additional permissions from that copy, or from any part of
it.  (Additional permissions may be written to require their own
removal in certain cases when you modify the work.)  You may place
additional permissions on material, added by you to a covered work,
for which you have or can give appropriate copyright permission.

  Notwithstanding any other provision of this License, for material you
add to a covered work, you may (if authorized by the copyright holders of
that material) supplement the terms of this License with terms:

    a) Disclaiming warranty or limiting liability differently from the
    terms of sections 15 and 16 of this License; or

    b) Requiring preservation of specified reasonable legal notices or
    author attributions in that material or in the Appropriate Legal
    Notices displayed by works containing it; or

    c) Prohibiting misrepresentation of the origin of that material, or
    requiring that modified versions of such material be marked in
    reasonable ways as different from the original version; or

    d) Limiting the use for publicity purposes of names of licensors or
    authors of the material; or

    e) Declining to grant rights under trademark law for use of some
    trade names, trademarks, or service marks; or

    f) Requiring indemnification of licensors and authors of that
    material by anyone who conveys the material (or modified versions of
    it) with contractual assumptions of liability to the recipient, for
    any liability that these contractual assumptions directly impose on
    those licensors and authors.

  All other non-permissive additional terms are considered "further
restrictions" within the meaning of section 10.  If the Program as you
received it, or any part of it, contains a notice stating that it is
governed by this License along with a term that is a further
restriction, you may remove that term.  If a license document contains
a further restriction but permits relicensing or conveying under this
License, you may add to a covered work material governed by the terms
of that license document, provided that the further restriction does
not survive such relicensing or conveying.

  If you add terms to a covered work in accord with this section, you
must place, in the relevant source files, a statement of the
additional terms that apply to those files, or a notice indicating
where to find the applicable terms.

  Additional terms, permissive or non-permissive, may be stated in the
form of a separately written license, or stated as exceptions;
the above requirements apply either way.

  8. Termination.

  You may not propagate or modify a covered work except as expressly
provided under this License.  Any attempt otherwise to propagate or
modify it is void, and will automatically terminate your rights under
this License (including any patent licenses granted under the third
paragraph of section 11).

  However, if you cease all violation of this License, then your
license from a particular copyright holder is reinstated (a)
provisionally, unless and until the copyright holder explicitly and
finally terminates your license, and (b) permanently, if the copyright
holder fails to notify you of the violation by some reasonable means
prior to 60 days after the cessation.

  Moreover, your license from a particular copyright holder is
reinstated permanently if the copyright holder notifies you of the
violation by some reasonable means, this is the first time you have
received notice of violation of this License (for any work) from that
copyright holder, and you cure the violation prior to 30 days after
your receipt of the notice.

  Termination of your rights under this section does not terminate the
licenses of parties who have received copies or rights from you under
this License.  If your rights have been terminated and not permanently
reinstated, you do not qualify to receive new licenses for the same
material under section 10.

  9. Acceptance Not Required for Having Copies.

  You are not required to accept this License in order to receive or
run a copy of the Program.  Ancillary propagation of a covered work
occurring solely as a consequence of using peer-to-peer transmission
to receive a copy likewise does not require acceptance.  However,
nothing other than this License grants you permission to propagate or
modify any covered work.  These actions infringe copyright if you do
not accept this License.  Therefore, by modifying or propagating a
covered work, you indicate your acceptance of this License to do so.

  10. Automatic Licensing of Downstream Recipients.

  Each time you convey a covered work, the recipient automatically
receives a license from the original licensors, to run, modify and
propagate that work, subject to this License.  You are not responsible
for enforcing compliance by third parties with this License.

  An "entity transaction" is a transaction transferring control of an
organization, or substantially all assets of one, or subdividing an
organization, or merging organizations.  If propagation of a covered
work results from an entity transaction, each party to that
transaction who receives a copy of the work also receives whatever
licenses to the work the party's predecessor in interest had or could
give under the previous paragraph, plus a right to possession of the
Corresponding Source of the work from the predecessor in interest, if
the predecessor has it or can get it with reasonable efforts.

  You may not impose any further restrictions on the exercise of the
rights granted or affirmed under this License.  For example, you may
not impose a license fee, royalty, or other charge for exercise of
rights granted under this License, and you may not initiate litigation
(including a cross-claim or counterclaim in a lawsuit) alleging that
any patent claim is infringed by making, using, selling, offering for
sale, or importing the Program or any portion of it.

  11. Patents.

  A "contributor" is a copyright holder who authorizes use under this
License of the Program or a work on which the Program is based.  The
work thus licensed is called the contributor's "contributor version".

  A contributor's "essential patent claims" are all patent claims
owned or controlled by the contributor, whether already acquired or
hereafter acquired, that would be infringed by some manner, permitted
by this License, of making, using, or selling its contributor version,
but do not include claims that would be infringed only as a
consequence of further modification of the contributor version.  For
purposes of this definition, "control" includes the right to grant
patent sublicenses in a manner consistent with the requirements of
this License.

  Each contributor grants you a non-exclusive, worldwide, royalty-free
patent license under the contributor's essential patent claims, to
make, use, sell, offer for sale, import and otherwise run, modify and
propagate the contents of its contributor version.

  In the following three paragraphs, a "patent license" is any express
agreement or commitment, however denominated, not to enforce a patent
(such as an express permission to practice a patent or covenant not to
sue for patent infringement).  To "grant" such a patent license to a
party means to make such an agreement or commitment not to enforce a
patent against the party.

  If you convey a covered work, knowingly relying on a patent license,
and the Corresponding Source of the work is not available for anyone
to copy, free of charge and under the terms of this License, through a
publicly available network server or other readily accessible means,
then you must either (1) cause the Corresponding Source to be so
available, or (2) arrange to deprive yourself of the benefit of the
patent license for this particular work, or (3) arrange, in a manner
consistent with the requirements of this License, to extend the patent
license to downstream recipients.  "Knowingly relying" means you have
actual knowledge that, but for the patent license, your conveying the
covered work in a country, or your recipient's use of the covered work
in a country, would infringe one or more identifiable patents in that
country that you have reason to believe are valid.

  If, pursuant to or in connection with a single transaction or
arrangement, you convey, or propagate by procuring conveyance of, a
covered work, and grant a patent license to some of the parties
receiving the covered work authorizing them to use, propagate, modify
or convey a specific copy of the covered work, then the patent license
you grant is automatically extended to all recipients of the covered
work and works based on it.

  A patent license is "discriminatory" if it does not include within
the scope of its coverage, prohibits the exercise of, or is
conditioned on the non-exercise of one or more of the rights that are
specifically granted under this License.  You may not convey a covered
work if you are a party to an arrangement with a third party that is
in the business of distributing software, under which you make payment
to the third party based on the extent of your activity of conveying
the work, and under which the third party grants, to any of the
parties who would receive the covered work from you, a discriminatory
patent license (a) in connection with copies of the covered work
conveyed by you (or copies made from those copies), or (b) primarily
for and in connection with specific products or compilations that
contain the covered work, unless you entered into that arrangement,
or that patent license was granted, prior to 28 March 2007.

  Nothing in this License shall be construed as excluding or limiting
any implied license or other defenses to infringement that may
otherwise be available to you under applicable patent law.

  12. No Surrender of Others' Freedom.

  If conditions are imposed on you (whether by court order, agreement or
otherwise) that contradict the conditions of this License, they do not
excuse you from the conditions of this License.  If you cannot convey a
covered work so as to satisfy simultaneously your obligations under this
License and any other pertinent obligations, then as a consequence you may
not convey it at all.  For example, if you agree to terms that obligate you
to collect a royalty for further conveying from those to whom you convey
the Program, the only way you could satisfy both those terms and this
License would be to refrain entirely from conveying the Program.

  13. Use with the GNU Affero General Public License.

  Notwithstanding any other provision of this License, you have
permission to link or combine any covered work with a work licensed
under version 3 of the GNU Affero General Public License into a single
combined work, and to convey the resulting work.  The terms of this
License will continue to apply to the part which is the covered work,
but the special requirements of the GNU Affero General Public License,
section 13, concerning interaction through a network will apply to the
combination as such.

  14. Revised Versions of this License.

  The Free Software Foundation may publish revised and/or new versions of
the GNU General Public License from time to time.  Such new versions will
be similar in spirit to the present version, but may differ in detail to
address new problems or concerns.

  Each version is given a distinguishing version number.  If the
Program specifies that a certain numbered version of the GNU General
Public License "or any later version" applies to it, you have the
option of following the terms and conditions either of that numbered
version or of any later version published by the Free Software
Foundation.  If the Program does not specify a version number of the
GNU General Public License, you may choose any version ever published
by the Free Software Foundation.

  If the Program specifies that a proxy can decide which future
versions of the GNU General Public License can be used, that proxy's
public statement of acceptance of a version permanently authorizes you
to choose that version for the Program.

  Later license versions may give you additional or different
permissions.  However, no additional obligations are imposed on any
author or copyright holder as a result of your choosing to follow a
later version.

  15. Disclaimer of Warranty.

  THERE IS NO WARRANTY FOR THE PROGRAM, TO THE EXTENT PERMITTED BY
APPLICABLE LAW.  EXCEPT WHEN OTHERWISE STATED IN WRITING THE COPYRIGHT
HOLDERS AND/OR OTHER PARTIES PROVIDE THE PROGRAM "AS IS" WITHOUT WARRANTY
OF ANY KIND, EITHER EXPRESSED OR IMPLIED, INCLUDING, BUT NOT LIMITED TO,
THE IMPLIED WARRANTIES OF MERCHANTABILITY AND FITNESS FOR A PARTICULAR
PURPOSE.  THE ENTIRE RISK AS TO THE QUALITY AND PERFORMANCE OF THE PROGRAM
IS WITH YOU.  SHOULD THE PROGRAM PROVE DEFECTIVE, YOU ASSUME THE COST OF
ALL NECESSARY SERVICING, REPAIR OR CORRECTION.

  16. Limitation of Liability.

  IN NO EVENT UNLESS REQUIRED BY APPLICABLE LAW OR AGREED TO IN WRITING
WILL ANY COPYRIGHT HOLDER, OR ANY OTHER PARTY WHO MODIFIES AND/OR CONVEYS
THE PROGRAM AS PERMITTED ABOVE, BE LIABLE TO YOU FOR DAMAGES, INCLUDING ANY
GENERAL, SPECIAL, INCIDENTAL OR CONSEQUENTIAL DAMAGES ARISING OUT OF THE
USE OR INABILITY TO USE THE PROGRAM (INCLUDING BUT NOT LIMITED TO LOSS OF
DATA OR DATA BEING RENDERED INACCURATE OR LOSSES SUSTAINED BY YOU OR THIRD
PARTIES OR A FAILURE OF THE PROGRAM TO OPERATE WITH ANY OTHER PROGRAMS),
EVEN IF SUCH HOLDER OR OTHER PARTY HAS BEEN ADVISED OF THE POSSIBILITY OF
SUCH DAMAGES.

  17. Interpretation of Sections 15 and 16.

  If the disclaimer of warranty and limitation of liability provided
above cannot be given local legal effect according to their terms,
reviewing courts shall apply local law that most closely approximates
an absolute waiver of all civil liability in connection with the
Program, unless a warranty or assumption of liability accompanies a
copy of the Program in return for a fee.

                     END OF TERMS AND CONDITIONS

            How to Apply These Terms to Your New Programs

  If you develop a new program, and you want it to be of the greatest
possible use to the public, the best way to achieve this is to make it
free software which everyone can redistribute and change under these terms.

  To do so, attach the following notices to the program.  It is safest
to attach them to the start of each source file to most effectively
state the exclusion of warranty; and each file should have at least
the "copyright" line and a pointer to where the full notice is found.

    <one line to give the program's name and a brief idea of what it does.>
    Copyright (C) <year>  <name of author>

    This program is free software: you can redistribute it and/or modify
    it under the terms of the GNU General Public License as published by
    the Free Software Foundation, either version 3 of the License, or
    (at your option) any later version.

    This program is distributed in the hope that it will be useful,
    but WITHOUT ANY WARRANTY; without even the implied warranty of
    MERCHANTABILITY or FITNESS FOR A PARTICULAR PURPOSE.  See the
    GNU General Public License for more details.

    You should have received a copy of the GNU General Public License
    along with this program.  If not, see <http://www.gnu.org/licenses/>.

Also add information on how to contact you by electronic and paper mail.

  If the program does terminal interaction, make it output a short
notice like this when it starts in an interactive mode:

    <program>  Copyright (C) <year>  <name of author>
    This program comes with ABSOLUTELY NO WARRANTY; for details type `show w'.
    This is free software, and you are welcome to redistribute it
    under certain conditions; type `show c' for details.

The hypothetical commands `show w' and `show c' should show the appropriate
parts of the General Public License.  Of course, your program's commands
might be different; for a GUI interface, you would use an "about box".

  You should also get your employer (if you work as a programmer) or school,
if any, to sign a "copyright disclaimer" for the program, if necessary.
For more information on this, and how to apply and follow the GNU GPL, see
<http://www.gnu.org/licenses/>.

  The GNU General Public License does not permit incorporating your program
into proprietary programs.  If your program is a subroutine library, you
may consider it more useful to permit linking proprietary applications with
the library.  If this is what you want to do, use the GNU Lesser General
Public License instead of this License.  But first, please read
<http://www.gnu.org/philosophy/why-not-lgpl.html>.
\end{verbatim}




\bibliography{../gpusph-manual}

\end{document}
