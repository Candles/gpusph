\documentclass{article}
\usepackage[paperheight=11cm,paperwidth=10cm,margin=0.1cm]{geometry}
\usepackage{amsmath}
\usepackage{amssymb}
\usepackage{asymptote}
\begin{document}
\begin{asy}
import math;
import patterns;
import geometry;

unitsize(2.5cm);

void drawpart(pair pos, int type, real radius, int id) {
	pen p;
	
	if (type == 0)
		p = green;
	else if (type == 1)
		p = orange;
	else if (type == 2)
		p = yellow;
	else
		p = currentpen;

	filldraw(circle(pos, radius), p, p);
	if (id >= 0)
		label(string(id), pos);
}

int size = 3;
pair p0 = (0, 0);
pair p1 = (size, size);

add(grid(size, size, black));

int[] sortedid = {6, 4, 21, 20, 27, 26, 28, 5, 12, 11, 14, 13, 15, 0, 1, 16, 17, 2, 3, 18, 19, 7, 8, 22, 23, 9, 10, 24, 25};
int id = 0;
drawpart((0, 0), 2, 0.2, sortedid[id]);
id += 1;
for (int i = 1; i <= 2*((int) size); i+=1) {
	pair pos = (0, 0.5*i);
	int type = ((i % 2) == 0 ) ? 2 : 1;
	drawpart(pos, type, 0.2, sortedid[id]);
	id += 1;
}
for (int i = 1; i <= 2*((int) size); i+=1) {
	pair pos = (0.5*i, 0);
	int type = ((i % 2) == 0 ) ? 2 : 1;
	drawpart(pos, type, 0.2, sortedid[id]);
	id += 1;
}

for (real x = 0.25; x <= 2; x += 0.5) {
	for (real y = 0.25; y <= 2; y += 0.5) {
		drawpart((x, y), 0, 0.2, sortedid[id]);
		id += 1;
	}
}

id = 0;
for (real y = 0.5; y < size; y += 1) {
	for (real x = 0.5; x <= size; x += 1) {
		pair pos = (x, y);
		label(string(id), pos, fontsize(16pt));
		id += 1;
	}
}

pair pos = (0, -0.4);
drawpart(pos, 0, 0.1, -1);
label("Fluid particle", pos + (0.1, 0) , E);
pos = (0, -0.7);
drawpart(pos, 1, 0.1, -1);
label("Boundary particle", pos + (0.1, 0) , E);
pos = (2, -0.4);
drawpart(pos, 2, 0.1, -1);
label("Vertex particle", pos + (0.1, 0) , E);

\end{asy}
\end{document}